\documentclass[a4paper,12pt,notitlepage]{article}

\usepackage{amsmath, amssymb, amsthm}
\usepackage{mathtools}
\usepackage[math]{fontspec}
\usepackage[nosingleletter, lastparline]{impnattypo}
\usepackage[polish]{babel}
\usepackage{bookmark}
\usepackage[margin=1in]{geometry}
\usepackage[babel=true, tracking=true]{microtype}
\usepackage{minted}
\usepackage{parskip}
\usepackage{array}
\usepackage{colortbl}
\usepackage{hhline}
\usepackage{subfig}
\usepackage{hhline}

\linespread{1.3}

\renewcommand*{\thesection}{\Alph{section}}
\setlength{\tabcolsep}{1.5mm}

\definecolor{bg}{rgb}{0.95,0.95,0.95}
\setminted{frame=single, bgcolor=bg, breaklines=true, autogobble}

\IfFontExistsTF{JetBrainsMono-Regular}{
    \setmonofont{JetBrainsMono}[
        UprightFont = *-Light,
        BoldFont = *-Regular,
        ItalicFont = *-Light-Italic,
        Scale = MatchLowercase
    ]
}{}

\title{\textbf{TD -- laboratorium 5}}
\author{Piotr Rogulski 305867 \\ Szymon Sieradzki 305881}
\date{\today}

\begin{document}

\maketitle

\section{Przydzielanie adresów IP}

\begin{table}[htbp]
    \centering
    \caption{Adresacja interfejsów}
    \subfloat[Ethernet]{\begin{tabular}{|*4c|}
        \hhline{|====|}
            \textbf{Router} & \textbf{Interfejs} & \textbf{Adres IP} & \textbf{Podsieć} \\
        \hline
            R1 & e0/0 & 10.0.12.1 & 10.0.12.0/30 \\
        \hline
            R2 & e0/0 & 10.0.12.2 & 10.0.12.0/30 \\
               & e0/1 & 10.0.24.1 & 10.0.24.0/30 \\
               & e0/2 & 10.0.23.1 & 10.0.23.0/30 \\
        \hline
            R3 & e0/0 & 10.0.35.1 & 10.0.35.0/30 \\
               & e0/2 & 10.0.23.2 & 10.0.23.0/30 \\
               & e0/3 & 10.0.34.1 & 10.0.34.0/30 \\
        \hline
            R4 & e0/1 & 10.0.24.2 & 10.0.24.0/30 \\
               & e0/2 & 10.0.45.1 & 10.0.45.0/30 \\
               & e0/3 & 10.0.34.2 & 10.0.34.0/30 \\
        \hline
            R5 & e0/0 & 10.0.35.2 & 10.0.35.0/30 \\
               & e0/2 & 10.0.45.2 & 10.0.45.0/30 \\
        \hhline{|====|}
    \end{tabular}}%
    \quad%
    \subfloat[Loopback]{\begin{tabular}{|*3c|}
        \hhline{|===|}
            \textbf{Router} & \textbf{Interfejs} & \textbf{Loopback IP} \\
        \hline
            R1 & L0 & 1.1.1.1/32      \\
               & L1 & 192.168.11.1/24 \\
        \hline
            R2 & L0 & 2.2.2.2/32      \\
               & L1 & 192.168.21.1/24 \\
        \hline
            R3 & L0 & 3.3.3.3/32      \\
               & L1 & 192.168.31.1/24 \\
        \hline
            R4 & L0 & 4.4.4.4/32      \\
               & L1 & 192.168.41.1/24 \\
               & L2 & 192.168.42.1/24 \\
               & L3 & 192.168.43.1/24 \\
        \hline
            R5 & L0 & 5.5.5.5/32      \\
               & L1 & 192.168.51.1/24 \\
        \hhline{|===|}
    \end{tabular}}
\end{table}

\section{Konfiguracja OSPF w AS 230}
System autonomiczny AS 230 zawiera 2 routery R2 i R3, by ich wzajemna komunikacja była możliwa konieczne jest skonfigurowanie protokołu OSPF dla tych routerów. W konfiguracji protokołu zostaje dodana podsieć R2-R3 tj. podsieć o adresie 10.0.23.0/30. Prawidłową konfigurację OSPF potwierdzają wpisy w tablicach routingu i rezultat polecenia  \mintinline{text}{ping <adres IP>} wykonanego z R2 na interfejs loopback R3 i odwrotnie.
\inputminted[label=Pingowanie interfejsu loopback R3 z R2, firstline=178, lastline=183]{text}{Routers/R2.txt}
\inputminted[label=Pingowanie interfejsu loopback R2 z R3, firstline=306, lastline=311]{text}{Routers/R3.txt}
\section{Podstawowa konfiguracja BGP}
\subsection{Sesja iBGP R2-R3}
Zainicjalizowanie sesji iBGP między routerami R2 i R3 następuje przez użycie poleceń  \mintinline{text}{router bgp 230} i  \mintinline{text}{neighbor <adres_sąsiada> remote-as 230}. Routery do połączenia wymagają ustawienia ich adresów jako adresów źródłowych przy pomocy  \mintinline{text}{neighbor <adres_sąsiada> update-source loopback 0}. Po konfiguracji sesja zostaje ustanowiona, co potwierdzają rezultaty polecenia  \mintinline{text}{show ip bgp neighbors} wywołanego na R2 i R3.
\inputminted[label=Stan sesji BGP po ustawieniu adresów źródłowychna R2, firstline=242, lastline=244]{text}{Routers/R2.txt}
\inputminted[label=Stan sesji BGP po ustawieniu adresów źródłowychna R3, firstline=439, lastline=441]{text}{Routers/R3.txt}
\subsection{Sesja eBGP R1-R2}
Między routerami R1 i R2 zainicjalizowana zostaje sesja eBGP przy pomocy poleceń użytych w poprzedniej sekcji z różnicą w podawanym numerze AS i bez ustawiania adresów źródłowych. Pomyślne ustanowienie eBGP potwierdzają rezultaty wywołania \mintinline{text}{show ip bgp summary} i \mintinline{text}{show ip bgp neighbors}.
\inputminted[label=Rezultat polecenia show ip bgp summary na R1, firstline=123, lastline=128]{text}{Routers/R1.txt}
\inputminted[label=Rezultat polecenia show ip bgp summary na R2, firstline=321, lastline=327]{text}{Routers/R2.txt}
\section{Zaawansowana konfiguracja BGP}

\end{document}
