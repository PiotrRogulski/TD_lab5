\documentclass[a4paper,12pt,notitlepage]{article}

\usepackage{amsmath, amssymb, amsthm}
\usepackage{mathtools}
\usepackage[math]{fontspec}
\usepackage[nosingleletter, lastparline]{impnattypo}
\usepackage[polish]{babel}
\usepackage{bookmark}
\usepackage[margin=1in]{geometry}
\usepackage[babel=true, tracking=true]{microtype}
\usepackage{minted}
\usepackage{parskip}
\usepackage{array}
\usepackage{colortbl}
\usepackage{hhline}
\usepackage{subfig}
\usepackage{hhline}

\linespread{1.3}

\renewcommand*{\thesection}{\Alph{section}}
\setlength{\tabcolsep}{1.5mm}

\definecolor{bg}{rgb}{0.95,0.95,0.95}
\setminted{frame=single, bgcolor=bg, breaklines=true, autogobble}

\IfFontExistsTF{JetBrainsMono-Regular}{
    \setmonofont{JetBrainsMono}[
        UprightFont = *-Light,
        BoldFont = *-Regular,
        ItalicFont = *-Light-Italic,
        Scale = MatchLowercase
    ]
}{}

\title{\textbf{TD -- laboratorium 5}}
\author{Piotr Rogulski 305867 \\ Szymon Sieradzki 305881}
\date{\today}

\begin{document}

\maketitle

\section{Przydzielanie adresów IP}

\begin{table}[htbp]
    \centering
    \caption{Adresacja interfejsów}
    \subfloat[Ethernet]{\begin{tabular}{|*4c|}
        \hhline{|====|}
            \textbf{Router} & \textbf{Interfejs} & \textbf{Adres IP} & \textbf{Podsieć} \\
        \hline
            R1 & e0/0 & 10.0.12.1 & 10.0.12.0/30 \\
        \hline
            R2 & e0/0 & 10.0.12.2 & 10.0.12.0/30 \\
               & e0/1 & 10.0.24.1 & 10.0.24.0/30 \\
               & e0/2 & 10.0.23.1 & 10.0.23.0/30 \\
        \hline
            R3 & e0/0 & 10.0.35.1 & 10.0.35.0/30 \\
               & e0/2 & 10.0.23.2 & 10.0.23.0/30 \\
               & e0/3 & 10.0.34.1 & 10.0.34.0/30 \\
        \hline
            R4 & e0/1 & 10.0.24.2 & 10.0.24.0/30 \\
               & e0/2 & 10.0.45.1 & 10.0.45.0/30 \\
               & e0/3 & 10.0.34.2 & 10.0.34.0/30 \\
        \hline
            R5 & e0/0 & 10.0.35.2 & 10.0.35.0/30 \\
               & e0/2 & 10.0.45.2 & 10.0.45.0/30 \\
        \hhline{|====|}
    \end{tabular}}%
    \quad%
    \subfloat[Loopback]{\begin{tabular}{|*3c|}
        \hhline{|===|}
            \textbf{Router} & \textbf{Interfejs} & \textbf{Loopback IP} \\
        \hline
            R1 & L0 & 1.1.1.1/32      \\
               & L1 & 192.168.11.1/24 \\
        \hline
            R2 & L0 & 2.2.2.2/32      \\
               & L1 & 192.168.21.1/24 \\
        \hline
            R3 & L0 & 3.3.3.3/32      \\
               & L1 & 192.168.31.1/24 \\
        \hline
            R4 & L0 & 4.4.4.4/32      \\
               & L1 & 192.168.41.1/24 \\
               & L2 & 192.168.42.1/24 \\
               & L3 & 192.168.43.1/24 \\
        \hline
            R5 & L0 & 5.5.5.5/32      \\
               & L1 & 192.168.51.1/24 \\
        \hhline{|===|}
    \end{tabular}}
\end{table}

\section{Konfiguracja OSPF w AS 230}

\section{Podstawowa konfiguracja BGP}

\section{Zaawansowana konfiguracja BGP}

\end{document}
