\documentclass[a4paper,12pt,notitlepage]{article}

\usepackage{amsmath, amssymb, amsthm}
\usepackage{mathtools}
\usepackage[math]{fontspec}
\usepackage[nosingleletter, lastparline]{impnattypo}
\usepackage[polish]{babel}
\usepackage{bookmark}
\usepackage[margin=1in]{geometry}
\usepackage[babel=true, tracking=true]{microtype}
\usepackage{minted}
\usepackage{parskip}
\usepackage{array}
\usepackage{colortbl}
\usepackage{hhline}
\usepackage{subfig}
\usepackage{hhline}

\linespread{1.3}

\renewcommand*{\thesection}{\Alph{section}}
\setlength{\tabcolsep}{1.5mm}

\definecolor{bg}{rgb}{0.95,0.95,0.95}
\setminted{frame=single, bgcolor=bg, breaklines=true, autogobble}

\IfFontExistsTF{JetBrainsMono-Regular}{
    \setmonofont{JetBrainsMono}[
        UprightFont = *-Light,
        BoldFont = *-Regular,
        ItalicFont = *-Light-Italic,
        Scale = MatchLowercase
    ]
}{}

\title{\textbf{TD -- laboratorium 5}}
\author{Piotr Rogulski 305867 \\ Szymon Sieradzki 305881}
\date{\today}

\begin{document}

\maketitle

\section{Przydzielanie adresów IP}

\begin{table}[htbp]
    \centering
    \caption{Adresacja interfejsów}
    \subfloat[Ethernet]{\begin{tabular}{|*4c|}
        \hhline{|====|}
            \textbf{Router} & \textbf{Interfejs} & \textbf{Adres IP} & \textbf{Podsieć} \\
        \hline
            R1 & e0/0 & 10.0.12.1 & 10.0.12.0/30 \\
        \hline
            R2 & e0/0 & 10.0.12.2 & 10.0.12.0/30 \\
               & e0/1 & 10.0.24.1 & 10.0.24.0/30 \\
               & e0/2 & 10.0.23.1 & 10.0.23.0/30 \\
        \hline
            R3 & e0/0 & 10.0.35.1 & 10.0.35.0/30 \\
               & e0/2 & 10.0.23.2 & 10.0.23.0/30 \\
               & e0/3 & 10.0.34.1 & 10.0.34.0/30 \\
        \hline
            R4 & e0/1 & 10.0.24.2 & 10.0.24.0/30 \\
               & e0/2 & 10.0.45.1 & 10.0.45.0/30 \\
               & e0/3 & 10.0.34.2 & 10.0.34.0/30 \\
        \hline
            R5 & e0/0 & 10.0.35.2 & 10.0.35.0/30 \\
               & e0/2 & 10.0.45.2 & 10.0.45.0/30 \\
        \hhline{|====|}
    \end{tabular}}%
    \quad%
    \subfloat[Loopback]{\begin{tabular}{|*3c|}
        \hhline{|===|}
            \textbf{Router} & \textbf{Interfejs} & \textbf{Loopback IP} \\
        \hline
            R1 & L0 & 1.1.1.1/32      \\
               & L1 & 192.168.11.1/24 \\
        \hline
            R2 & L0 & 2.2.2.2/32      \\
               & L1 & 192.168.21.1/24 \\
        \hline
            R3 & L0 & 3.3.3.3/32      \\
               & L1 & 192.168.31.1/24 \\
        \hline
            R4 & L0 & 4.4.4.4/32      \\
               & L1 & 192.168.41.1/24 \\
               & L2 & 192.168.42.1/24 \\
               & L3 & 192.168.43.1/24 \\
        \hline
            R5 & L0 & 5.5.5.5/32      \\
               & L1 & 192.168.51.1/24 \\
        \hhline{|===|}
    \end{tabular}}
\end{table}

\section{Konfiguracja OSPF w AS 230}

System autonomiczny AS 230 zawiera 2 routery R2 i R3, zatem by ich wzajemna komunikacja była możliwa konieczne jest skonfigurowanie protokołu OSPF dla tych routerów. W konfiguracji protokołu zostaje dodana podsieć R2-R3, tj. podsieć o adresie 10.0.23.0/30. Prawidłową konfigurację OSPF potwierdzają wpisy w tablicach routingu i rezultat polecenia \mintinline{text}{ping <adres IP>} wykonanego z R2 na interfejs loopback R3 i odwrotnie.
\inputminted[label=Pingowanie interfejsu loopback R3 z R2, firstline=178, lastline=183]{text}{Routers/R2.txt}%
\inputminted[label=Pingowanie interfejsu loopback R2 z R3, firstline=306, lastline=311]{text}{Routers/R3.txt}%

\section{Podstawowa konfiguracja BGP}

\subsection{Sesja iBGP R2-R3}

Zainicjalizowanie sesji iBGP między routerami R2 i R3 następuje przez użycie poleceń \mintinline{text}{router bgp 230} i \mintinline{text}{neighbor <adres_sąsiada> remote-as 230}. Routery do połączenia wymagają ustawienia ich adresów jako adresów źródłowych przy pomocy komendy \mintinline{text}{neighbor <adres_sąsiada> update-source loopback 0}. Po konfiguracji sesja zostaje ustanowiona, co potwierdzają rezultaty polecenia \mintinline{text}{show ip bgp neighbors} wywołanego na R2 i R3.
\inputminted[label=Stan sesji BGP po ustawieniu adresów źródłowych na R2, firstline=242, lastline=244]{text}{Routers/R2.txt}%
\inputminted[label=Stan sesji BGP po ustawieniu adresów źródłowych na R3, firstline=439, lastline=441]{text}{Routers/R3.txt}%

\subsection{Sesja eBGP R1-R2}

Między routerami R1 i R2 zainicjalizowana zostaje sesja eBGP przy pomocy poleceń użytych w poprzedniej sekcji z różnicą w podawanym numerze AS i bez ustawiania adresów źródłowych. Pomyślne ustanowienie eBGP potwierdzają rezultaty wywołania \mintinline{text}{show ip bgp summary} i \mintinline{text}{show ip bgp neighbors}.
\inputminted[label=Rezultat polecenia show ip bgp summary na R1, firstline=123, lastline=128]{text}{Routers/R1.txt}%
\inputminted[label=Rezultat polecenia show ip bgp summary na R2, firstline=321, lastline=327]{text}{Routers/R2.txt}%

\subsection{Rozgłaszanie podsieci w BGP}

Rozgłaszanie podsieci za pośrednictwem protokołu BGP odbywa się za pomocą polecenia \mintinline{text}{network <adres> mask <maska>}. Rozgłoszona zostaje podsieć reprezentowana przez interfejs loopback 1 routera R1, tj. podsieć o adresie 192.168.11.1/24. Analiza wpisów w tablicach routingu na routerach R2 i R3 pokazuje, że podsieć R1 jest widoczna w routerze R2 i nie jest widoczna w R3. Wynika to z faktu, że podsieć jest rozgłaszana tylko w eBGP ustanowionym między R1 i R2.
\inputminted[label= Tablica routingu na R2, firstline=468, lastline=496]{text}{Routers/R2.txt}%
\inputminted[label= Tablica routingu na R3, firstline=528, lastline=547]{text}{Routers/R3.txt}%

By w tablicy routingu R3 widoczna była podsieć R1 należy ustawić atrybut next-hop rozgłaszany z R2 do R3 był ustawiony na router R2. Odbywa się to przy pomocy polecenia \mintinline{text}{neighbor <adres> next-hop-self}. Adres podsieci jest teraz obecny w tablicy routingu R2. W dalszym ciągu jednak polecenie \mintinline{text}{ping} z R3 na adres R1 kończy się niepowodzeniem. Wynika to z jednostronnej komunikacji między routerami, podczas gdy do pomyślnego zakończenia polecenia potrzebne jest połączenie dwukierunkowe (informacja zwrotna o powodzeniu). Rozwiązaniem tego problemu jest rozgłoszenie adresu podsieci R2-R3.
\inputminted[label= Pingowanie adresu R1 z routera R3, firstline=582, lastline=587]{text}{Routers/R3.txt}%

\subsection{Konfiguracja pozostałych sesji eBGP}

W dalszej części skonfigurowane zostają pozostałe sesje eBGP między systemami autonomicznymi. Odbywa się to w sposób analogiczny dla ustanawiania sesji między R1 i R2. Ustanowione zostają połączenia R2-R4, R3-R4, R3-R5, R4-R5. Analizując wyniki polecenia \mintinline{text}{show ip bgp summary} na routerze R3 można stwierdzić, że sesje zostały ustanowione pomyślnie.
\inputminted[label=Rezultat polecenia show ip bgp summary na R3, firstline=806, lastline=821]{text}{Routers/R3.txt}%

\subsection{Rozgłaszanie adresów loopback}

Rozgłoszone zostają teraz adresy loopback L1, L2 i L3 w sposób analogiczny do adresu loopback L1 routera R1. Analizując tablicę routingu na routerze R3 można stwierdzić, że wszystkie adresy pomyślnie się rozprzestrzeniły.
\inputminted[label=Tablica routingu na R3 po rozgłoszeniu loopbacków, firstline=891, lastline=916]{text}{Routers/R3.txt}%

\subsection{Zmiana trasy pomiędzy R1 a R5}

Przeanalizowana zostanie teraz komunikacja routerów R1 i R5. W tym celu na routerze R1 zostaną wykonane polecenia \mintinline{text}{ping} i \mintinline{text}{traceroute} z adresem źródłowym ustawionym na loopback 1 i adresem loopback 1 z R5 jako adresem docelowym. Pomyślne pingowanie potwierdza komunikację między routerami. Obierana przez dane trasa to R2-R4-R5. Długości tras R2-R4-R5 i R2-R3-R5 są takie same, dlatego wybrana została ścieżka z routerem o wyższym numerze.
\inputminted[label=Komunikacja między R1 i R5, firstline=316, lastline=330]{text}{Routers/R1.txt}%

By zmusić router do obrania trasy R2-R3-R5 można nadpisać atrybut next-hop na R2 przez R3. Odbywa się to w sposób przedstawiony w sekcji C.3. Ponowne wywołanie traceroute pokazuje, że dane obierają zmienioną trasę.
\inputminted[label=Komunikacja między R1 i R5 po zmianie next-hop, firstline=331, lastline=338]{text}{Routers/R1.txt}%

\section{Zaawansowana konfiguracja BGP}

\subsection{Konfiguracja local preference}

Bez dodatkowej konfiguracji pakiety z R1 do R4 idą najkrótszą ścieżką, tj. R1-R2-R4.

\inputminted[label=Komunikacja między R1 i R4 bez local preference, firstline=339, lastline=345]{text}{Routers/R1.txt}%

AS 230 zostanie skonfigurowany tak, aby cały ruch wysyłany z R1 do R4 opuszczał ten system przez interfejs e0/3 routera R3.

\inputminted[label=Ustawianie local preference na R3, firstline=931, lastline=935]{text}{Routers/R3.txt}%

Aby zaobserwować zmiany należy wyczycić sesję BGP pomiędzy R3 a R4 poleceniem \mintinline{text}{clear ip bgp 10.0.34.2 soft} na routerze R3. Po zastosowaniu konfiguracji pakiety z R1 do R4 przechodzą przez połączenie R3-R4.

\inputminted[label=Komunikacja między R1 i R4 z local preference, firstline=346, lastline=353]{text}{Routers/R1.txt}%

\subsection{Path prepending}

Aby ruch przychodzący do podsieci loopback1 routera R5 wchodził do AS 500 przez AS 400, należy użyć mechanizmu path prepending. W tym celu tworzy się route-mapę, która dla ruchu wychodzącego wstawia na początek AS-path wielokrotnie numer systemu AS 500. Następnie utworzoną route-mapę należy dodać do połączenia R5-R3 w sesji BGP:

\inputminted[label=Konfiguracja path prepending dla interfejsu R5-R3, firstline=372, lastline=378]{text}{Routers/R5_2.txt}%

Po skonfigurowaniu pakiety z R1 do R5 idą ścieżką R1-R2-R4-R5.

\inputminted[label=Komunikacja między R1 i R5 z path prepending, firstline=533, lastline=540]{text}{Routers/R1.txt}%

\inputminted[label=baza danych BGP na R1, firstline=557, lastline=572]{text}{Routers/R1.txt}%

Po wyłączeniu interfejsu R4-R5 pakiety z powrotem idą przez router R3:

\inputminted[label=Komunikacja między R1 i R5 z path prepending, firstline=573, lastline=580]{text}{Routers/R1.txt}%

\end{document}
